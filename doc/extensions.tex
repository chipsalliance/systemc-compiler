
\subsection{Advanced FIFO}\label{section:adv_fifo}

Advanced FIFO is a collection of FIFO modules with push and pop interfaces. Advanced FIFO is intended for interaction between two processes which can be called producer and consumer. The FIFO is used to store pushed request from producer when consumer is not ready to pop. Each of the processes can be method or thread, they may be located in the same or in different modules.

Advanced FIFO is implemented as a SystemC modules with signal and function interfaces. There are the following modules in {\tt sct\_fifo.h}:

\begin{itemize}
\item adv\_fifo\_base -- base class for all FIFO modules, not intended to be used;
\item adv\_fifo -- normal FIFO with signal interface;
\item adv\_fifo\_mif -- normal FIFO with function interface, uses adv\_fifo inside;
\item mcp\_request\_fifo -- MCP (multi-clock path) FIFO for producer operating at lower frequency; 
\item mcp\_response\_fifo -- MCP FIFO for consumer operating at lower frequency.
\end{itemize}

FIFO signal interface has the following inputs/outputs:
%
\begin{lstlisting}[style=mycpp]
sc_in_clk      clk;              // Common clock, positive edge is used
sc_in<bool>    nrst;             // Asynchronous reset, low active
sc_in<bool>    push;             // Push data into FIFO
sc_in<T>       data_in;          // Input data
sc_out<bool>   ready_to_push;    // Ready to push
sc_in<bool>    pop;              // Pop data from FIFO
sc_out<T>      data_out;         // Output data
sc_out<bool>   out_valid;        // Output data is valid 
sc_out<bool>   almost_full;      // FIFO is almost full, number of free slots 
                                 // equal or less than AFULL_ELEMENT_NUM 
\end{lstlisting}

FIFO function interface has the following methods:
%
\begin{lstlisting}[style=mycpp]
    // FIFO is ready to push
    bool ready();
    // Push or clear push, push is ignored if FIFO is not ready to push
    // \return ready to push flag
    bool push(const T& data, bool push = true);
    // FIFO output data is valid, pop return can be used only if data is valid
    bool valid();
    // Pop or get data from FIFO, do not remove data from FIFO if pop = false
    T pop(bool pop = true);
    // FIFO is almost full, there is AFULL_ELEMENT_NUM elements or more used
    bool full();
    // Add FIFO signals to sensitivity list
    void addTo(sc_sensitive& s);
    // Bind FIFO clock and reset
    template <typename CLK_t, typename RSTN_t>
    void clk_nrst(CLK_t& clk_in, RSTN_t& nrst_in);
\end{lstlisting}

Advanced FIFO module has template parameters which specify FIFO size and other parameters:  
%
\begin{lstlisting}[style=mycpp]
template<
    typename T,                    // FIFO slot data type 
    bool ASYNC_VALID,              // Assert out_valid combinationally
    bool ASYNC_READY,              // Assert ready_to_push combinationally
    bool ASYNC_AFULL,              // Assert almost_full combinationally
    unsigned FIFO_LENGTH,          // Number of FIFO slots
    unsigned AFULL_ELEMENT_NUM,    // Number of free element slots in FIFO
                                   // when almost_full is asserted
    bool INIT_BUFFER               // Initialize FIFO slots in reset with zeros
>
class adv_fifo : public sc_module {...}
\end{lstlisting}

The FIFO {\tt push} and {\tt pop} may be asserted whenever. FIFO does push operation when both {\tt push}/{\tt ready\_to\_push} asserted. FIFO does pop operation when both {\tt pop}/{\tt out\_valid} asserted.
 
Outputs {\tt out\_valid}, {\tt ready\_to\_push} and {\tt almost\_full} may be asserted combinationally, de-assertion of them is done synchronously only. 

The FIFO allows to pop an element in the same clock it is pushed into empty FIFO, if {\tt ASYNC\_VALID} is true ({\tt out\_valid} is asserted combinationally).

The FIFO allows to push an element into full FIFO if there is pop operation in the same clock, if {\tt ASYNC\_READY} is true ({\tt ready\_to\_push} is asserted combinationally).

The FIFO provides combinational assertion of {\tt almost\_full} when FIFO is one element lower than {\tt AFULL\_ELEMENT\_NUM} and there is push and no pop. This feature is not MCP ready.


\subsection{SystemVerilog intrinsic insertion}\label{section:black_box}

This section describe how to insert SystemVerilog intrinsic ("black box") module.

ICSC supports replacement a SystemC module with given SystemVerilog intrinsic module. In this case no parsing of the SystemC module is performed, so this module can contain non-synthesizable code. To replace SystemC module it needs to define {\tt \_\_SC\_TOOL\_VERILOG\_MOD\_\_} variable of {\tt std::string} type in the module body.
{\tt \_\_SC\_TOOL\_VERILOG\_MOD\_\_} value can be specified in place or in the module constructor.

There are two common usages:
\begin{enumerate}
\item Replace with given SystemVerilog module: \_\_SC\_TOOL\_VERILOG\_MOD\_\_ contains SV module code or {\tt \#include} directive;
\item Do not generate module at all: \_\_SC\_TOOL\_VERILOG\_MOD\_\_ is empty string. 
\end{enumerate}
%
In second case SystemVerilog module implementation needs to be provided in an external file.

\begin{lstlisting}[style=mycpp]
struct my_register : sc_module {
  std::string __SC_TOOL_VERILOG_MOD__[] = R"(
     module my_register (
        input  logic [31:0] din,
        output logic [31:0] dout
     );
     assign dout = din;
     endmodule)";

  SC_CTOR (my_register) {...}
  ...
}
\end{lstlisting}
%
\begin{lstlisting}[style=mycpp]
// SystemVerilog generated
// Verilog intrinsic for module: my_register 
module my_register (
    input  logic [31:0] din,
    output logic [31:0] dout
);
assign dout = din;
endmodule
\end{lstlisting}



\subsection{Memory module name}

This section describes how to create a custom memory module with module name specified. 

To support vendor memory it needs to specify memory module name at instantiation point and exclude the SV module code generation (memory module is external one). To exclude SV module code generation empty {\tt \_\_SC\_TOOL\_VERILOG\_MOD\_\_} should be used. To specify memory module name it needs to define {\tt \_\_SC\_TOOL\_MODULE\_NAME\_\_} variable in the module body and initialize it with required name string.

If there are two instances of the same SystemC module, it is possible to give them different names, but {\tt \_\_SC\_TOOL\_VERILOG\_MOD\_\_} must be declared in the module. If {\tt \_\_SC\_TOOL\_VERILOG\_MOD\_\_} is not declared the SystemC module, only one SV module with first given name will be generated . 

Module name could be specified for module with non-empty {\tt \_\_SC\_TOOL\_VERILOG\_MOD\_\_}, but module names in {\tt \_\_SC\_TOOL\_MODULE\_NAME\_\_} and {\tt \_\_SC\_TOOL\_VERILOG\_MOD\_\_} should be the same.

If specified module name in module without {\tt \_\_SC\_TOOL\_VERILOG\_MOD\_\_} declaration conflicts with another module name, it updated with numeric suffix. Specified name in module with {\tt \_\_SC\_TOOL\_VERILOG\_MOD\_\_}  declaration never changed, so name uniqueness should be checked by user.

\begin{lstlisting}[style=mycpp]
// Memory stub example
struct memory_stub : sc_module {
    // Disable Verilog module generation
    std::string __SC_TOOL_VERILOG_MOD__[] = "";  
    // Specify module name at instantiation
    std::string __SC_TOOL_MODULE_NAME__;             
    explicit memory_stub(const sc_module_name& name,
                         const char* verilogName = "") :
        __SC_TOOL_MODULE_NAME__(verilogName)
    {}
};

// Memory instance at some module
memory_stub  stubInst1{"stubInst1", "pxxxrf256x32ben"};
memory_stub  stubInst2{"stubInst2", "pxxxsram1024x32ben"};
memory_stub  stubInst3{"stubInst3"};
stubInst1.clk(clk); 
stubInst2.clk(clk); 
stubInst3.clk(clk); 
...
\end{lstlisting}
%
\begin{lstlisting}[style=mycpp]
// SystemVerilog generated 
pxxxrf256x32ben     stubInst1(.clk(clk), ...);
pxxxsram1024x32ben  stubInst2(.clk(clk), ...);
memory_stub         stubInst3(.clk(clk), ...);
\end{lstlisting}


\section{Advanced verification features}\label{section:assertions}

\subsection{Immediate assertions}

There are several types of C++, SystemC, and ICSC assertions to use in design verification:

\begin{itemize}
\item assert(expr) -- general C++ assertion, in case of violation leads to abort SystemC simulation, ignored by ICSC;
\item sc\_assert(expr) -– SystemC assertion, leads to report fatal error (SC\_REPORT\_FATAL), ignored by ICSC;
\item sct\_assert(expr [, msg = ""]) -– ICSC assertion, in simulation has the same behavior as assert, SVA generates System Verilog assertion (SVA) for it. Second parameter const char* msg is optional, contains message to print in simulation and used in SVA error message;
\end{itemize}

Immediate assertions are declared in {\tt sct\_assert.h} ({\tt include/sct\_common/sct\_assert.h}). This assertion can be used for SystemC simulation ans well as for generated Verilog simulation. In generated Verilog there is equivalent SVA assert with error message if specified. Error message should be string literal ({\tt const char*}).

\begin{lstlisting}[style=mycpp]
// SystemC source
#include "sct_assert.h"
void sct_assert_method()  {
   sct_assert(cntr == 1);
   sct_assert(!enable, "User error message");
}
\end{lstlisting}
%
\begin{lstlisting}[style=myverilog]
// Generated SystemVerilog
assert (cntr == 1) else $error("Assertion failed at test_sva_assert.cpp:55:9");
assert (!enable) else $error("User error message at test_sva_assert.cpp:56:9");
\end{lstlisting}

SVA for {\tt sct\_assert} generated in {\tt always\_comb} block that requires to consider exact delta cycle when used in the assertion signals/ports changed their values. That makes using {\tt sct\_assert} more complicated than temporal assertion {\tt SCT\_ASSERT} which described below. So, it is strongly recommended to use {\tt SCT\_ASSERT(expr, clk.pos())} instead of {\tt sct\_assert(expr)}.

\subsection{Temporal assertions}

Temporal assertions in SystemC intended to be used for advanced verification of design properties with specified delays. These assertions looks similar to System Verilog assertions (SVA). The assertions can be added in SystemC design in module scope and clocked thread process:

\begin{lstlisting}[style=mycpp]
SCT_ASSERT(EXPR, EVENT);                      // In module scope 
SCT_ASSERT(LHS, TIME, RHS, EVENT);            // In module scope 
SCT_ASSERT_THREAD(LHS, TIME, RHS, EVENT);     // In clocked thread 
SCT_ASSERT_LOOP(LHS, TIME, RHS, EVENT, ITER); // In for-loop inside of clocked thread
\end{lstlisting}

These ways are complementary. Assertions in module scope avoids polluting process code. Assertions in clock thread allows to use member and local variables. Assertions in loop can access channel and port arrays.

Temporal assertions in module scope and clocked thread have the same parameters:
\begin{itemize}
\item EXPR -- assertion expression, checked to be true,
\item LHS -- antecedent assertion expression which is pre-condition, 
\item TIME -- temporal condition is specific number of cycles or cycle interval,
\item RHS -- consequent assertion expression, checked to be true if antecedent expression was true in past,
\item EVENT -- cycle event which is clock positive, negative or both edges.
\item ITER -- loop iteration counter variable(s) in arbitrary order.
\end{itemize}

If {\tt clk} is clock input, then EVENT specified with {\tt clk.pos()}, {\tt clk.neg()} or {\tt clk} correspondingly. 

Assertion expression can be arithmetical or logical expression, with zero, one or several operands. Assertion expression cannot contain function call and ternary operator {\tt ?}.

Temporal condition specified with:
\begin{lstlisting}[style=mycpp]
SCT_TIME(TIME) -- time delay, TIME is number of cycles,
SCT_TIME(LO_TIME, HI_TIME) -- time interval in number of cycles.
\end{lstlisting}

Temporal condition specifies time delay when RHS checked after LHS is true. Temporal condition is number of cycles or cycle interval, where cycle is clock period. Specific number of cycles is integer non-negative number. Cycle interval has low time and high time, each of them is integer non-negative number. Low time and high time can be the same. There is reduced form of time condition with brackets only.

Temporal assertions are declared in {\tt sct\_assert.h} ({\tt include/sct\_common/sct\_assert.h}), it needs to be included. 

To disable temporal assertions macro {\tt SCT\_ASSERT\_OFF} should be defined. That can be required to use another HLS tools which does not support these assertions.
To avoid SVA assertion generating {\tt NO\_SVA\_GENERATE} option of {\tt svc\_target} should be used. 

\subsubsection{Temporal assertions in module scope}

Temporal assertions in module scope added with 

\begin{lstlisting}[style=mycpp]
SCT_ASSERT(EXPR, EVENT);
SCT_ASSERT (LHS, TIME, RHS, EVENT);
\end{lstlisting}
%
Assertion expression can operate with signals, ports, template parameters, constants and literals. Member data variables (not signals/ports) access in assertion leads to data race and therefore not supported. 

There are several examples:
\begin{lstlisting}[style=mycpp]
static const unsigned N = 3;
sc_in<bool> req;
sc_out<bool> resp;
sc_signal<sc_uint<8>> val;
sc_signal<sc_uint<8>>* pval;
int m;
sc_uint<16> arr[N];
...
SCT_ASSERT(req || val == 0, clk.pos());             // OK
SCT_ASSERT(req, SCT_TIME(1), resp, clk.pos());      // OK
SCT_ASSERT(req, SCT_TIME(N+1), resp, clk.neg());    // OK
SCT_ASSERT(req, (2), val.read(), clk);              // OK, brackets only form
SCT_ASSERT(val, SCT_TIME(2,3), *pval, clk.pos());   // OK, time interval
SCT_ASSERT(arr[0], (N,2*N), arr[N-1], clk.pos());   // OK, brackets only form
SCT_ASSERT(val == N, (1), resp, clk.pos());         // OK, constant in assertion expression 
SCT_ASSERT(m == 0, (1), resp, clk.pos());           // Error, member data variable used
SCT_ASSERT(resp, (0,2), arr[m+1], clk.pos());       // Error, access array at non-literal index
\end{lstlisting}
%
Generated SVA:
\begin{lstlisting}[style=myverilog]
`ifndef INTEL_SVA_OFF
sctAssertLine80 : assert property (
    @(posedge clk) true |-> req || val == 0 );
sctAssertLine81 : assert property (
    @(posedge clk) req |=> resp );
sctAssertLine82 : assert property (
    @(negedge clk) req |-> ##4 resp );
sctAssertLine83 : assert property (
    @(negedge clk) req |-> ##2 val );
...
`endif // INTEL_SVA_OFF
\end{lstlisting}

\subsubsection{Temporal assertions in clocked thread process}

Temporal assertions in clocked thread added with 
\begin{lstlisting}[style=mycpp]
SCT_ASSERT_THREAD (LHS, TIME, RHS, EVENT);
\end{lstlisting}
%
These assertions can operate with local data variables and local/member constants. They also can operate with module member data variables which are modified in this process.

Assertion in thread process can be placed in reset section (before first {\tt wait()}) or after reset section before main infinite loop. Assertions in main loop not supported. Assertions can be placed in {\tt if} branch scopes, but this {\tt if} must have statically evaluated condition. Variable condition of assertion should be considered in its antecedent (left) expression. 

\begin{lstlisting}[style=mycpp]
void thread_proc() {
   // Reset section
   ...
   SCT_ASSERT_THREAD(req, SCT_TIME(1), ready, clk.pos());  // Assertion in reset section
   wait();                        
   SCT_ASSERT_THREAD(req, SCT_TIME(2,3), resp, clk.pos()); // Assertion after reset

   // Main loop 
   while (true) { 
      ...                                   // No assertion in main loop 
      wait();
}}
\end{lstlisting}

Assertion in reset section generated in the end of {\tt always\_ff} block, that makes it active under reset. Assertion after reset section generated in else branch of the reset if, that makes it inactive under reset.

\begin{lstlisting}[style=myverilog]
// Generated Verilog code
always_ff @(posedge clk or negedge nrst) begin
   if (~nrst) begin
      ...
   end else 
   begin 
      ... 
      assert property (req |-> ##[2:3] resp);  // Assertion after reset section 
   end 
   assert property (req |=> ready);   // Assertion from reset  
end
\end{lstlisting}

There an example with several assertions:

\begin{lstlisting}[style=mycpp]
static const unsigned N = 3;
sc_in<bool> req;
sc_out<bool> resp;
sc_signal<bool> resp;
sc_uint<8> m;
...
void thread_proc() {
   int i = 0;
   SCT_ASSERT_THREAD(req, SCT_TIME(0), ready, clk.pos());     // OK
   SCT_ASSERT_THREAD(req, SCT_TIME(N+1), ready, clk.pos());   // OK 
   SCT_ASSERT_THREAD(req, (2,3), i == 0, clk.pos());          // OK, local variable used
   wait();
   if (N > 1) {
       SCT_ASSERT_THREAD(req, SCT_TIME(1), resp, clk.pos());  // OK, statically evaluated
   }
   SCT_ASSERT_THREAD(m > 1, (2), ready, clk.pos())            // OK, member variable used
   while (true) {   
      ...
      SCT_ASSERT_THREAD(req, SCT_TIME(0), ready, clk.pos());  // Error
      wait();
}}
\end{lstlisting}

\subsubsection{Temporal assertions in loop inside of clocked thread}

Temporal assertions in loop inside of clocked thread added with 
\begin{lstlisting}[style=mycpp]
SCT_ASSERT_LOOP (LHS, TIME, RHS, EVENT, ITER);
\end{lstlisting}
%
ITER parameter is loop variable name or multiple names separated by comma.

Loop with assertions can be in reset section or after reset section before main infinite loop. The loop should be {\tt for}-loop with statically determined number of iteration and one counter variable. Such loop cannot have {\tt wait()} in its body. 

\begin{lstlisting}[style=mycpp]
void thread_proc() {
   // Reset section
   ...
   for (int i = 0; i < N; ++i) {
      SCT_ASSERT_LOOP(req[i], SCT_TIME(1), ready[i], clk.pos(), i);
      for (int j = 0; j < M; ++j) {
         SCT_ASSERT_LOOP(req[i][j], SCT_TIME(2), resp[i][N-j+1], clk.pos(), i, j);
   }}
   wait();                        
   while (true) { 
      ...                        // No assertion in main loop 
      wait();
}}
\end{lstlisting}
%
\begin{lstlisting}[style=myverilog]
// Generated Verilog code
always_ff @(posedge clk or negedge nrst) begin
   if (~nrst) begin
      ...
   end else 
   begin 
      ... 
   end 
   for (integer i = 0; i < N; i++) begin
       assert property ( req[i] |=> ready[i] );  
   end 
   for (integer i = 0; i < N; i++) begin
       for (integer j = 0; j < M; j++) begin
           assert property ( req[i][j] |-> ##2 resp[i][M-j+1] );  
       end
   end 
end
\end{lstlisting}

\subsection{Special assertions}\label{section:assert_special}

There are special assertions mostly intended for tool developers.
\begin{itemize}
\item sct\_assert\_latch(var [, latch = true]) -- assert that given variable, signal or port is latch if second parameter is true (by default), or not latch otherwise. Latch object is defined only at some paths of method process.
\item sct\_assert\_const(expr) -- check given expression is true in constant propagation analysis
\item sct\_assert\_level(level) -- check current block level with given one
\item sct\_assert\_unknown(value) -- check give value is unknown, i.e. not statically evaluated
\item sct\_assert\_defined(expr) -- check given expression  is defined 
\item sct\_assert\_read(expr) -- check given expression is read 
\item sct\_assert\_register(expr) -- check given expression is read before defined 
\item sct\_assert\_array\_defined(expr) -- check given expression is array and some element is defined at least on some paths
\end{itemize}


